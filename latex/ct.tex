\documentclass[article]{aaltoseries}
\usepackage[utf8]{inputenc}
\usepackage{tikz-cd} 
\usepackage{amssymb}
\usepackage{listings}
\usepackage{color}

\definecolor{identifierColor}{rgb}{0.65,0.16,0.16}

\lstset{
  language=haskell,
  aboveskip=3mm,
  belowskip=3mm,
  showstringspaces=false,
  columns=flexible,
  basicstyle={\small\ttfamily},
  numbers=none,
  numberstyle=\tiny\color{gray},
  keywordstyle=\color{identifierColor},
  commentstyle=\color{dkgreen},
  stringstyle=\color{mauve},
  breaklines=true,
  breakatwhitespace=true,
  tabsize=2,
  numbersep=5pt,
  showspaces=false,
}

\begin{document}
 
%=========================================================

\title{Applications of Category Theory in Web programming}

\author{Patrik Vilja Peter Vilja} % Do _not_ add your student number

\affiliation{\texttt{patrik.vilja@aalto.fi} \texttt{peter.vilja@aalto.fi}}

\maketitle

%==========================================================

\begin{abstract}
  Both category theory and Web applications have recently gained more
  popularity. Category theory have been applied in many fields, such as in
  design of functional programming languages, such as Haskell and PureScript,
  and the Web programming trend is moving towards purely functional programming
  languages. This paper introduces the basic concepts of the category theory
  such as a definition of a category, functor and natural transformation. Web
  programming concepts are reviewed from a categorical point of view. Complete
  examples of how the concepts of category theory can be used to ease the
  development and reasoning of Web based programs are presented and discussed.

\vspace{3mm}
\noindent KEYWORDS: category theory, category, morphism, composition, monad,
functor, natural transfromation, web programming, purescript 

\end{abstract}


%============================================================

\section{Introduction}

  Category theory is a fairly recent field in mathematics introduced by S.
  Eilenberg and S. Mac Lane in 1945 \cite{awodey2006category,
    eilenberg1945general}. Concepts of category theory can be used in different
  fields of mathematics and it provides a way to describe difficult problems
  from an abstract point of view \cite{eilenberg1945general, mclarty2007last}.
  It has applications in multiple fields such as in programming and
  in separate fields of mathematics \cite{awodey2006category, jones1993glasgow,
    reynolds1980using}.

  Category theory has emerged from the urge to abstract over structure. We
  humans interpret the world around us in terms of structures and patterns.
  Category theory aims to generalize these structures and provide high-level
  abstraction over them \cite{eilenberg1945general}. In general, category theory
  hides the details of a structure thus similarities with different branches of
  mathematics have been found \cite{lawvere1996adjointness}.

  One of the core principles around category theory is composability
  \cite{barr1990category}. With the rule of composability we can split a problem
  into smaller understandable subproblems which can then be composed together to
  solve the bigger problem. A category consists of objects and morphisms.
  Objects and morphisms alone does not contain any details of a category, but
  the composition of morphisms between objects describe the category
  \cite{barr1990category, computational}. A composition of morphisms also has to
  fulfill certain laws described later in this paper. In essence, any structure
  that implements composability and identity can be seen as a category.
  Additionally to a definition of a category, functors and natural transformations
  are the main principles of category theory \cite{awodey2006category,
    computational}. Commonly a category is described with commutative diagram
  that contains only objects and morphisms.
 
  This paper introduces how concepts of category theory can be used in Web
  programming although some examples are not limited only to the Web programming.
  The paper reviews basic operations in Web such as asynchronous execution, DOM
  manipulation and user interaction from a categorical point of view with
  concrete examples in PureScript. Advantages of category theory in Web
  programming are weighted with the focus on simplicity and reasonability of a
  program. Furthermore, subjects for future work are discussed.
 
  The remainder of this paper is organized as follows. Section 2 introduces the
  main concepts of category theory. Section 3 describes how these concepts can
  be used in Web programming and Section 4 weights the results. Finally Section
  6 presents conclusions.

%============================================================

\section{Category theory}
Category theory consists of general concepts which can be applied across
different fields of mathematics and in programming. These concepts are abstract
and can present various abstract constructions. This section reviews
basic concepts of category theory and shortly provides examples of existing
implementations of these concepts in programming languages.

%------------------------------------------------------------

  \subsection{Category}
  A category consists of objects and morphisms, also known as arrows, between
  each pair of objects in the category \cite{barr1990category,
    mac2013categories}. Morphisms has to have a source and a target called
  \textbf{domain} (dom) and \textbf{codomain} (cod) \cite{awodey2006category,
    barr1990category, mac2013categories}. They describe how objects are
  connected in a category and there can be any number of morphisms between two
  objects \cite{barr1990category, computational}. For example, two objects can
  be connected with an infinite amount of morphisms or there could be no morphisms
  between the objects. Morphisms with domain $a$ and codomain $b$ in a category
  $C$ belong to a \textbf{hom-set} $hom_C(a, b)$ \cite{barr1990category,
    mac2013categories}. A category can contain infinite amount of objects and
  morphisms. In category theory, morphisms between objects describe the category
  thus there is no restrictions to the internal structure of objects in a
  category \cite{computational}. Identity, composability and abstraction
  are important features of a category \cite{awodey2006category,
    barr1990category}. A category can be thought as a graph with an identity
  and composition properties, where morphisms (edges) connects objects (nodes)
  \cite{barr1990category}.

%------------------------------------------------------------

  \subsubsection{Identity}
    For every object $a$ in a category $C$ there has to exist an identity
    morphism, also known as $id_a: a \rightarrow a$ \cite{mac2013categories,
      computational}. For identity morphism domain and codomain are equal $dom(a)
    = a = cod(a)$ \cite{mac2013categories}. Identity morphism $id_a$ acts as an
    identity when morphisms are composed. Additionally, for every morphism $f$,
    in a category $C$, there has to exist Left and Right identity. Thus we can
    compose identity morphism to both sides of a composition with another
    morphism. In Left identity we compose a morphism $f: a \rightarrow b$ with
    an identity morphism $id_b$ such that $id_b \circ f = f$
    \cite{barr1990category, computational}. The identity morphism $id_b$ has no
    effect on the resulting composed morphism and thus $id_b \circ f$ and $f$
    are equal. Similarly in Right identity we compose a morphism $g: a
    \rightarrow b$ with an identity morphism $id_a$ such that $g \circ id_a = g$
    \cite{barr1990category, computational}. Additionally the identity morphism
    $id_a$ has no effect therefore $g \circ id_a$ and $g$ are equal.
 
    An identity element can be stated with Left and Right identity as follows.
    $f \circ id_a = f = id_b \circ f$ where $f: a \rightarrow b$, $id_a: a
    \rightarrow a$ and $id_b: b \rightarrow b$ \cite{barr1990category, computational}.

%------------------------------------------------------------

  \subsubsection{Composability}
    Composition is a property of a category \cite{barr1990category,
      mac2013categories}. A category is described by compositions of morphisms
    as objects and morphisms are primitives and do not contain any details of a
    category. Morphisms need to be composable and the composition of morphisms
    has to fulfill the laws of associativity \cite{barr1990category,
      mac2013categories, computational}. Each pair of morphisms $f$ and $g$ with
    $cod(f) = dom(g)$ must have a composition $g \circ f$ where $f: a
    \rightarrow b$, $g: b \rightarrow c$ and $g \circ f: a \rightarrow c$
    \cite{mac2013categories}. As morphisms need to be associative $h \circ (g
    \circ f)$ must always be equal with $(h \circ g) \circ f$ where $f$ and $g$
    are same as before and $h: c \rightarrow d$ \cite{awodey2006category,
      barr1990category, eilenberg1945general, mac2013categories, computational}.
 
    $Set_{Rel}$, for example, is a category which contains sets as
    objects and relations between sets as morphisms where a relation is a subset
    of cartesian product of two sets \cite{computational}. Also another slightly
    different category is just a Set category where objects are sets and morphisms
    are total functions \cite{barr1990category, computational}. A total function
    is a function which is defined for all possible arguments. In programming, a
    category is usually presented using types and morphisms with functions
    \cite{computational}.

% ------------------------------------------------------------

  \subsubsection{Monomorphism, Epimorphism, Isomorphism}
    In Set theory there exists properties of functions such as injection,
    surjection and bijection. These properties can also be defined in terms of
    category theory and they are defined for any category \cite{barr1990category}.

    A morphism $f: B \rightarrow C$ in category $D$ is a \textbf{monomorphism}
    if $\forall A.\,\forall g, h: A \rightarrow B.\,f \circ g = f \circ h
    \Rightarrow g = h$ \cite{barr1990category, pierce1991basic}. In other words,
    a morphism $f$ is a monomorphism if any two morphisms from any type $A$ to
    $B$ composed with morphism $f$ are equal. Monomorphism is categorical
    analogue to injective function in the Set theory in a category of Set
    \cite{barr1990category, pierce1991basic}.

    A morphism $f: A \rightarrow B$ in category $D$ is an \textbf{epimorphism}
    if $\forall C.\,\forall g, h: B \rightarrow C.\,g \circ f = h \circ f
    \Rightarrow g = h$ \cite{barr1990category, pierce1991basic}. In other words,
    a morphism $f$ is an epimorphism if morphism $f$ composed with any two
    morphisms from $B$ to any type $C$ are equal. Epimorphism is categorical
    analogue to surjective function in the Set theory in a category of Set
    \cite{barr1990category, pierce1991basic}.

    A morphism $f: A \rightarrow B$ in category $D$ is an \textbf{isomorphism}
    if there exists an inverse morphism $f^{-1}: B \rightarrow A$ and
    compositions $f^{-1} \circ f$ and $f \circ f^{-1}$ equals $id_A$ and $id_B$
    respectively \cite{barr1990category, pierce1991basic}. If isomorphism exists
    between objects $A$ and $B$, they are \textbf{isomorphic}. Hereby, these two
    objects are equal up to isomorphism \cite{pierce1991basic}. Commutative diagram
    \ref{isomorphism} presents isomorphism between objects $a$ and $b$ in terms
    of morphisms $f$, $f^{-1}$, $id_A$ and $id_B$.

    \begin{equation}
      \label{isomorphism}
      \begin{tikzcd}[row sep=huge, column sep=huge]
        a \arrow[r, bend left, "f"]
        \arrow[loop above, distance=2em, in=60, out=120, "id_a"]
        & b \arrow[l, bend left, "f^{-1}"]
        \arrow[loop above, distance=2em, in=60, out=120, "id_b"]
      \end{tikzcd}
    \end{equation}

% ------------------------------------------------------------

  \subsection{Universal construction} 
    A mathematical construction has some properties that defines a structure.
    Universal construction is used to define properties of structures in terms
    of category theory. These properties are called as universal properties. As
    category theory abstracts the details, all reasoning of universal properties
    of some structure has to be done based on morphisms and their compositions.
 
    Usually a category contains multiple objects that fulfill some properties and
    structure. Universal construction picks the best candidate among these
    objects. For every other candidate of object that holds the universal
    property there exists an unique morphism from the candidate to the object that
    has the universal property. In other word, the object that satisfies the given
    universal property is an initial or terminal object in the category of all
    candidates \cite{pierce1991basic}.
  
% ------------------------------------------------------------

  \subsubsection{Initial object}
    An object $a$ in category $C$ is \textbf{an initial object} if $\forall b \in
    C.\,\exists f: a \rightarrow b$ (for every object $b$ in category $C$ there
    exists a morphism $f$ from initial object $a$ to the object $b$) and every
    two morphisms $f: a \rightarrow b$, $g: a \rightarrow b$ have to be
    equal \cite{barr1990category, pierce1991basic}. In other words, there exists
    exactly one morphism from $a$ to $b$. Commutative diagram \ref{initial} of an initial
    object $a$ can be found below. A category can have multiple initial
    objects. In category of Set the initial object is an universal construction
    of an empty set $\emptyset$ in terms of category theory and it is the only initial
    object for every Set \cite{barr1990category, pierce1991basic}.

    \begin{equation}
      \label{initial}
      \begin{tikzcd}[row sep=huge, column sep=huge]
        & b \\
        a \arrow[ur, "absurd"]
        \arrow[loop right, distance=2em, "id_{a}"]
        & 
      \end{tikzcd}
    \end{equation}

% ------------------------------------------------------------

  \subsubsection{Terminal object}
    Terminal object of a category is the dual of initial object. An object $a$
    in category $C$ is \textbf{a terminal object} if $\forall b \in C.\,\exists
    f: b \rightarrow a$ (for every object $b$ in category $C$ there exists a
    morphism $f$ from the object $b$ to the terminal object $a$) and for every
    two morphisms $f: b \rightarrow a$, $g: b \rightarrow a$ have to be equal
    \cite{barr1990category, pierce1991basic}. In other words, there exists
    exactly one morphism from $b$ to $a$. Commutative diagram \ref{terminal} of
    a terminal object $a$ can be found below. A category can have multiple terminal
    objects. In category of Set the terminal object is an universal construction
    of singleton set in terms of category theory. Furthermore, each singleton
    set is a terminal object \cite{barr1990category, pierce1991basic}.

    \begin{equation}
      \label{terminal}
      \begin{tikzcd}[row sep=huge, column sep=huge]
        & a \arrow[loop right, distance=2em, "id_{a}"] \\
        b \arrow[ur, "unit_b"]
        & Void \arrow[u, "unit_{void}"']
      \end{tikzcd}
    \end{equation}
   
% ------------------------------------------------------------

  \subsection{Algebraic data types}
    Algebraic data types such as product and coproduct defines a foundation to
    structures such as tuples and records. Product and coproduct data types
    obeys algebraic laws of identity, associativity and commutativity up to
    isomorphism respectively \cite{barr1990category}. These two data types are
    widely used in programming languages.

% ------------------------------------------------------------

  \subsubsection{Product}
    In category theory a product is denoted as a cartesian product of a pair of
    objects. A product of objects $A$ and $B$ is an object $A \times B$. The
    object $A \times B$ has two projections $p : A \times B \rightarrow A$ and
    $q : A \times B \rightarrow B$. In addition, universal construction defines
    that there exists an unique morphism $m$ from every other candidate of $A
    \times B$ object called $C$ to object $A \times B$ such that $p \circ m =
    p'$ and $q \circ m = q'$. Where $p' : C \rightarrow A$ and $q' : C
    \rightarrow B$ \cite{barr1990category, pierce1991basic}. In
    other words, universal construction defines the $A \times B$ object in terms
    of all other objects in the category that have morphisms to $A$ and $B$
    objects thus object $A \times B$ is a terminal object in a category of all
    product candidates. Morphisms and objects of product can be found below in
    form of commutative diagram \ref{product}.

    \begin{equation}
      \label{product}
      \begin{tikzcd}[row sep=huge, column sep=huge]
        & C \arrow[d, dashed, "m"]
        \arrow[ddl, "p'"']
        \arrow[ddr, "q'"]
        & \\
        & A \times B \arrow[dl, "p"]
        \arrow[dr, "q"']
        & \\
        [-2em] A
        &
        & B
      \end{tikzcd}
    \end{equation}

    Product is usually presented as a tuple in most programming languages and it
    is denoted with parenthesis and comma. For example \lstinline|(a, b)| is a
    common presentation of tuple. Product type is called an algebraic data type
    as it behaves like a multiplication and it obeys algebraic laws up to
    isomorphism. Product is symmetric up to isomorphism, types
    \lstinline|(a, b)| and \lstinline|(b, a)| are not the same but they preserve
    the same structure. An isomorphism between \lstinline|(a, b)| and \lstinline|(b, a)|
    is commonly defined with a function called \lstinline|swap| and a product is
    also associative up to isomorphism. Product type also has an unit type
    \lstinline|()| up to isomorphism and thus it obeys algebraic laws of monoid
    \cite{barr1990category}.

% ------------------------------------------------------------

  \subsubsection{Coproduct}
    Coproduct is the dual of the product. It is also known as tagged union,
    disjoint union, sum type or variant. The coproduct can be constructed by
    inverting the morphisms of the product category. The coproduct is denoted as
    $A \amalg B$ or $A + B$. The object $A \amalg B$ is constructed from two
    injection morphisms $i : A \rightarrow A \amalg B$ and $j : B \rightarrow A
    \amalg B$ that belongs to category $C$. In addition, universal construction
    defines that there exists an unique morphism $m$ from $A \amalg B$ to every
    other candidate of $A \amalg B$ called $C$, such that $m \circ i = i'$ and
    $m \circ j = j'$. Where $i' : A \rightarrow C$ and $j' : B \rightarrow C$
    \cite{barr1990category, pierce1991basic}. Object $A \amalg B$ is an initial
    object in a category of all coproduct candidates \cite{pierce1991basic}.
    Morphisms and objects of coproduct can be found below in form of commutative
    diagram \ref{coproduct}.

    \begin{equation}
      \label{coproduct}
      \begin{tikzcd}[row sep=huge, column sep=huge]
        A \arrow[dr, "i"]
        \arrow[ddr, "i'"']
        &
        & B \arrow[dl, "j"']
        \arrow[ddl, "j'"] \\
        [-2em] & A \amalg B \arrow[d, dashed, "m"]
        & \\
        & C
        &
      \end{tikzcd}
    \end{equation}

    The coproduct is usually presented with sum types. Coproduct obeys algebraic
    laws up to isomorphism and it can be seen as a sum. The coproduct is
    symmetric, associative and it has unit up to isomorphism \cite{barr1990category}.
    
% ------------------------------------------------------------

  \subsection{Functor}
    Functors are mappings, also known as morphisms, between categories
    \cite{barr1990category, mac2013categories, computational}. A functor $F: A
    \rightarrow B$ preserves the structure of a category by having a mapping
    from objects of a category $A$ to objects of a category $B$ ($F: a
    \rightarrow F\, a$) and a mapping from morphisms of $A$ to morphisms of $B$
    ($F: (f: a \rightarrow b) \rightarrow F\, f: F\, a \rightarrow F\, b$). The
    mapping of objects and morphisms are presented in commutative diagram
    \ref{functor}. The mappings must satisfy following rules to preserve
    composition and identity morphisms.

    \begin{itemize}
      \item $F(g \circ f) = F\, g \circ F\, f$, when a category $A$ contains any
        composition of any morphisms $g$ and $f$, $g \circ f$.
      \item $F\, id_a = id_{F\, a}$
    \end{itemize}

    \begin{equation}
      \label{functor}
      \begin{tikzcd}[row sep=huge, column sep=huge]
        & a \arrow[d, "f"] \arrow[dl, "g\, \circ\, f"] \arrow[r, "F"] \arrow[loop, distance=2em, in=60, out=120, "id_a"]
        &[5em] F\, a \arrow[d, "F\, f"'] \arrow[dr, "F\, (g\, \circ\, f)"]
                   \arrow[loop, distance=2em, in=60, out=120, "F\, id_a"]
        & \\
        c
        & b \arrow[l, "g"] \arrow[r, "F"]
        & F\, b \arrow[r, "F\, g"]
        & F\, c \\
      \end{tikzcd}
    \end{equation}

    Therefore, a functor needs to preserve all connections between objects from
    the source category, although there is no requirement for the resulting
    target category to obtain the same amount of morphisms thus multiple
    morphisms can be presented as a single morphism \cite{barr1990category,
      computational}. More precisely a functor is called as a faithful
    functor when the function between $hom_A(a, b)$ and $hom_B(F\, a, F\, b)$ for
    each $a, b \in A$ is injective. Similarly, a functor is called as a full
    functor when the function between $hom_A(a, b)$ and $hom_B(F\, a, F\, b)$ for
    each $a, b \in A$ is surjective. A functor can also be full and faithful at
    the same time.
 
    Additionally a functor $F: A \rightarrow A$, which maps objects and
    morphisms of a category $A$ to the same category, is called an
    \textbf{endofunctor} \cite{barr1990category, computational}. In programming
    languages functors are actually endofunctors. A functor can be composed to a
    new functor, for example functors $F: A \rightarrow B$ and $G: B \rightarrow
    C$ can be composed to a functor $G \circ F: A \rightarrow C$. A simple
    example of a functor is a \lstinline|Maybe| functor which maps objects such
    that \lstinline|a -> Maybe a| and morphisms with a function
    \lstinline|map :: (a -> b) -> Maybe a -> Maybe b|.

% ------------------------------------------------------------

  \subsection{Natural transformation}
    Natural transformation is a mapping between functors \cite{barr1990category,
      mac2013categories, computational}. Whereas functors mapped objects and
    morphisms between two categories a natural transformation maps the resulting
    objects from functors together. Thus for two functors $F: A
    \rightarrow B$ and $G: A \rightarrow B$ a natural transformation
    $\alpha: F \rightarrow G$ maps each object in a category $A$ a morphism
    between resulting objects $F\, a$ and $G\, a$ in $B$ from hom-set $hom_B(F\, a,
    G\, a)$. More formally $\alpha_a: F\, a \rightarrow G\, a$. Additionally
    natural transformation $\alpha$ has to follow naturality condition where for
    every morphism $f: a \rightarrow b$ in a category $A$ there has to exist equal
    compositions $\alpha_b \circ F\, f = G\, f \circ \alpha_a$
    \cite{barr1990category, eilenberg1945general, computational}. That is, for
    every two objects $a$ and $b$ in a category $A$ which are connected with at
    least a single morphism we have the following commutative diagram
    \ref{natural} with a naturality square that needs to commute.
    
    \begin{equation}
      \label{natural}
      \begin{tikzcd}[column sep=huge]
        a \arrow[r, "F"] \arrow[d, "f"] \arrow[ddr, "G"]
        &[7em] F\, a \arrow[dd, "\alpha_a"] \arrow[dr, "F\, f"]
        & \\
        b \arrow[rr, "F"] \arrow[ddrr, "G"]
        &
        & F\, b \arrow[dd, "\alpha_b"] \\
        [-1.5em]
        & G\, a \arrow[dr, "G\, f"]
        & \\
        [0.5em]
        &
        & G\, b \\
      \end{tikzcd}
    \end{equation}

    Similarly to a functor a natural transformation has to preserve the
    structure of a category. By definition the structure is preserved by
    preserving morphism between the object $F\, a \rightarrow G\, a$ and having a
    naturality square of every morphism between two objects in the source
    category. Additionally, as functors, a natural transformation can collapse
    morphisms from a category $A$ to a single morphism in category $B$. Natural
    transformation can also be thought as a family of morphisms between functors.
 
    In programming natural transformation is a polymorphic function, a function
    which is parameterised by a type. For example two functors \lstinline|List|
    and \lstinline|Maybe| transform objects and morphisms from a category to
    another category and the natural transformation is the mapping between these
    two functors. A natural transformation
    \lstinline|safeHead :: [a] -> Maybe a| is a natural transformation between
    \lstinline|List| and \lstinline|Maybe| functors that safely picks the first
    element of a list. The naturality condition can be presented as
    \lstinline|safeHead <<< fmap f (x:xs) = fmap f <<< safeHead (x:xs)|.

% ------------------------------------------------------------
    
  \subsection{Monoid}
    Monoids are semigroups with identity morphism. A monoid is a category with a
    single object and at least an identity morphism \cite{awodey2006category,
      barr1990category, mac2013categories, yorgey2012monoids}. Any morphisms
    in a monoid can be composed together as the result is always the same
    object. Thus morphisms in a monoid are endomorphisms which are morphisms
    from an object to itself. Monoid has to have an associative binary operator,
    for example multiplication $(a * b) * c = a * (b * c)$. Additionally a
    monoid has an unit element $\exists u\,\forall a.\, B(a,u) = B(u,a) = a$ ($B$
    is the binary operator) which does not have any effect in the binary
    operation. For example, if the binary operator is a multiplication function
    which operates on integers a unit element is 1.

% ------------------------------------------------------------

  \subsection{Monad}
    Monad, also known as triple, is all about composition. It is an endofunctor
    with two natural transformations $\eta : I_T \rightarrow T$ and $\mu : T^2
    \rightarrow T$ where $T$ is an endofunctor in an arbitrary category $C$,
    $T^2$ is a composition of a endofunctor with itself $T \circ T$ and $I_T$ is
    an identity functor \cite{barr1990category, moggi1989computational}.
    Therefore $\eta : I_T \rightarrow T$ means that $\eta$ is a natural
    transformation from identity functor to an endofunctor $T$ and $\mu : T^2
    \rightarrow T$ means that $\mu$ is a natural transformation from composition
    of two endofunctors $T$ to an endofunctor $T$. Similarly to monoid $\eta$ is
    an unit and $\mu$ is a multiplication transformation. Commutative diagram
    \ref{eta} presents a natural tranformation $\eta$ and diagram \ref{mu} a natural
    transformation $\mu$.
    
    \begin{equation}
      \label{eta}
      \begin{tikzcd}[column sep=huge]
        C \arrow[r, bend left, "I_T", ""{name=A, below}]
        \arrow[r, bend right, "T"{below}, ""{name=B}]
        & C
        \arrow[Rightarrow, from=A, to=B, "\eta"{right=4}]
      \end{tikzcd}
    \end{equation}

    \begin{equation}
      \label{mu}
      \begin{tikzcd}[column sep=huge]
        C \arrow[r, bend left, "T"]
        \arrow[rr, bend right=30, "T^2", ""{name=A, below}]
        \arrow[rr, bend right=100, "T"{below}, ""{name=B}]
        & C
        \arrow[r, bend left, "T"]
        & C
        \arrow[Rightarrow, from=A, to=B, "\mu"{right=4}]
      \end{tikzcd}
    \end{equation}

    A monad has to satisfy identity laws. Left identity $I_T \circ T = T$, where
    $T$ can be reached from $I_T \circ T$ with a natural transformation
    $\mu \circ (\eta \circ T)$, and right identity $T \circ I_T = T$ where $T$
    can be reached from $T \circ I_T$ with a natural transformation $\mu \circ
    (T \circ \eta)$ \cite{mac2013categories, moggi1989computational}.
    Commutative diagram \ref{identity} presents left and right identity laws.
    Additionally a monad has to satisfy the law of associativity $\mu \circ (\mu
    \circ T) = \mu \circ (T \circ \mu)$. When $T$ is composed with a natural
    transformation such as $T \circ \mu$ the $T$ presents identity natural
    transformation $I_T$ instead of endofunctor. Commutative
    diagram \ref{associativity} presents the associativity law which monad needs
    to fulfill.

    \begin{equation}
      \label{identity}
      \begin{tikzcd}[row sep=huge, column sep=huge]
        I_T\, \circ\, T
        \arrow[r, "\eta\, \circ\, T"]
        \arrow[dr, equal]
        & T^2
        \arrow[d, "\mu"]
        & T\, \circ\, I_T
        \arrow[l, "T\, \circ\, \eta"{above}]
        \arrow[dl, equal] \\
        {}
        & T
        &
      \end{tikzcd}
    \end{equation}

    \begin{equation}
      \label{associativity}
      \begin{tikzcd}[row sep=huge, column sep=huge]
        T^3 \arrow[r, "\mu\, \circ\, T"]
        \arrow[d, "T\, \circ\, \mu"]
        & T^2 \arrow[d, "\mu"] \\
        T^2 \arrow[r, "\mu"]
        & T
      \end{tikzcd}
    \end{equation}

    In Haskell natural transformations $\eta$ and $\mu$ are known as
    \lstinline|return| and \lstinline|join| and in PureScript they are known as
    \lstinline|pure| and \lstinline|join|. \lstinline|Maybe|,
    \lstinline|Either|, \lstinline|IO|, \lstinline|Reader|, \lstinline|Writer|
    and \lstinline|State| are examples of popular monads used in programming.


%============================================================

\section{Applications}

Basic functional programming patterns use concepts of category theory under the
hood. Categorical structures and patterns can be found all over Web
programming. For example, in Web applications many operations, such as logging,
DOM manipulation and XMLHttpRequests are achieved with side-effects. These
impure actions can be encapsulated with concepts of category theory. Product and
sum types, in turn, provide a structure to the data and patterns such as monads and
functors enhance the data flow. This chapter presents how category
theory can be applied in Web programming with PureScript.

%------------------------------------------------------------
  \subsection{Product and Coproduct}
    In PureScript a product is defined as \lstinline|data Tuple a b| which has
    two projections to extract data \lstinline|fst| and \lstinline|snd|. Product
    can also be defined as a record and in general all records are just
    extensions of a categorical product. An isomorphism between
    \lstinline|Tuple a b| and \lstinline|Tuple b a| is defined with a function
    called \lstinline|swap|.
    
    In PureScript canonical presentation of the coproduct is the sum type
    \lstinline{data Either a b = Left a | Right b}. Where constructors
    \lstinline|Left| and \lstinline|Right| correspond to coproduct injections.
    All sum types can be seen as an extension of the coproduct. The value of a
    sum type can be extracted with pattern matching.
    
    With product and coproduct we can construct algebraic properties as data
    types. For instance, distributive property of multiplication $x\,*\,(y\,+\,z) =
    x\,*\,y\,+\,x\,*\,z$ can be presented with algebraic data types as
    \lstinline|Product x (Either y z) ~ Either (Product x y) (Product x z)| up
    to isomorphism.

    Moreover, almost every data structure can be defined with the help of
    product and coproduct. For example a data type \lstinline|List|, defined as
    \lstinline{data List a = Nil | Cons a (List a)}, uses both product and
    coproduct types. Data constructor \lstinline|Cons| is an analogue to product
    type and \lstinline|List| is a coproduct of either \lstinline|Nil| or \lstinline|Cons|.

    
%------------------------------------------------------------

  \subsection{Functors}
    Endofunctors such as \lstinline|Maybe| and \lstinline|Either| have many use
    cases in Web programming. As a functor \lstinline|Maybe| maps values from a
    category to either \lstinline|Nothing| or \lstinline|Just| and is
    implemented simply as \lstinline{data Maybe a = Nothing | Just a}. Thus a
    \lstinline|Maybe| is an ideal structure to describe existence of a value. In
    Web programming values, such as user inputs, and features, such as HTML5
    functions, might or might not exist which makes Web programs unsafe.
    \lstinline|Maybe| can be used to describe the existence of a value and to enforce
    proper error handling. For example picking the first element of an empty
    list produces an undefined value which is likely to cause unwanted behavior
    later in the execution chain. The first value of a \lstinline|List| can be
    presented with the help of \lstinline|Maybe| endofunctor to enforce error
    handling. A function \lstinline|safeHead :: Array a -> Maybe a| maps
    \lstinline|a| to a type \lstinline|Maybe a|, which is an example of mapping
    object from a category to a category of functors with an endofunctor as in
    diagram \ref{mapping}.

    \begin{equation}
      \label{mapping}
      \begin{tikzcd}[row sep=huge, column sep=huge]
        a \arrow[r, "Maybe"]
        & Maybe\, a
      \end{tikzcd}
    \end{equation}

    \lstinline|safeHead| function can be used for example when displaying the
    first user from a list of users with function composition
    \lstinline|safeHead <<< getUsers|, where \lstinline|getUsers| is defined as
    \lstinline|getUsers :: () -> [User]|. Thus \lstinline|safeHead| is a mapping
    between two categories \lstinline|User -> Maybe User| with \lstinline|Maybe|
    endofunctor. The new type \lstinline|Maybe User| enforces to handle a case
    where an empty array was returned from \lstinline|getUsers| and makes it
    easier to reason about the program.

    Previous examples have presented mapping of objects to objects inside
    another category, but also morphisms are mapped. Diagram \ref{lifting}
    presents how a morphism can be mapped with Maybe functor.

    \begin{equation}
      \label{lifting}
      \begin{tikzcd}[row sep=huge, column sep=huge]
        a \arrow[r, "Maybe"]
        \arrow[d, "f"]
        & Maybe\, a
        \arrow[d, "map\, f"]
        \\
        b \arrow[r, "Maybe"]
        & Maybe\, b
      \end{tikzcd}
    \end{equation}

    Consider an Web application with a search field and a morphism
    \lstinline|getUserByName :: String -> User|. When the search field is
    submitted the \lstinline|getUserByName| function will be invoked and returns
    a user. As before the user input can be mapped to \lstinline|Maybe|,
    \lstinline|userInput :: Maybe String|, as there might not be any user input.
    Now that the user input is of type \lstinline|Maybe String| the
    \lstinline|getUserByName| cannot be used as such, but as functors need to
    preserve structure a morphism can be lifted with
    \lstinline|map :: Functor f => (a -> b) -> f a -> f b| and applied to the
    \lstinline|userInput| as follows. 

    \begin{lstlisting}[language=Haskell]
      user :: Maybe User
      user = map getUserByName userInput
    \end{lstlisting}

    Furthermore additional morphisms, such as getting the name of the user, can
    be mapped by just lifting more functions with map as mapping an endofunctor
    returns the same endofunctor.
 
    In addition to mapping of objects and morphisms, a functor needs to preserve
    identity and composition morphisms thus a function composition of any two
    functions \lstinline|a| and \lstinline|b|, \lstinline|a <<< b|, can be
    mapped multiple ways. For example, composition of functions
    \lstinline|toUpper| and \lstinline|trim| can be mapped as
    \lstinline|map toUpper <<< map trim| or \lstinline|map (toUpper <<< trim)|.
    Functor laws provide a powerful tool to reason and refactor application code
    as functions can be interchanged based on the laws.

    In Web programming there often exists a requirement to communicate with
    backend service(s). As a result responses from external services need to be
    parsed in order to further process the data, the parsing might fail if the
    given message was invalid. This can be avoided by mapping the message to
    another category with \lstinline|Either| functor. Consider a case where JSON
    string is parsed to further process the data with a mapping from
    \lstinline|String -> Either String Json|. Functor \lstinline|Either|
    contains either an error message of type \lstinline|String| or a
    successfully parsed message of type \lstinline|Json|. Again knowing that
    \lstinline|Either| is a functor, it follows certain laws and morphisms can
    be lifted with \lstinline|map| function.


%------------------------------------------------------------

  \subsection{Natural Transformation}
    As shown before functors can be used to reason and structure Web programs
    but in order to compose functors there has to exist mappings between them.
    Natural transformation can be used to transform a functor to a another
    functor. Previously mapping of objects was presented with a function
    \lstinline|safeHead :: Array a -> Maybe a|. If the function’s type signature
    is changed to \lstinline|safeHead :: List a ~> Maybe a| there exists a
    natural transformation between functors \lstinline|List| and
    \lstinline|Maybe|. In categorical terms an object \lstinline|a| is mapped to
    \lstinline|List a| with \lstinline|List| functor and to \lstinline|Maybe a| with
    \lstinline|Maybe| functor and there exists a natural transformation between
    \lstinline|List a| and \lstinline|Maybe a|.

    Consider another example of natural transformation where data is searched
    with endofunctor \lstinline|Maybe String| and asynchronous effect functor
    \lstinline|Aff| which encapsulates the search request. In order to execute
    the search request a natural transformation from
    \lstinline|Maybe String -> Aff e a| has to exist.

%------------------------------------------------------------

  \subsection{Monads}
    Function composition is a powerful tool to reason programs. Embellished
    functions, which return a value wrapped inside an endofunctor, can be
    composed with monads. The previously presented function
    \lstinline|getUserByName :: String -> User| was used to showcase mapping of
    morphisms with a functor. To be precise the type signature of the
    \lstinline|getUserByName| function needs to be changed to
    \lstinline|getUserByName :: String -> Maybe User| as a user might not be
    found with every possible input value. Using this new function would lead
    into a following scenario.
 
    \begin{lstlisting}[language=Haskell]
      user :: Maybe (Maybe User)
      user = map getUserByName userInput
    \end{lstlisting}
 
    The user value would be a \lstinline|Maybe User| wrapped inside a
    \lstinline|Maybe|. In category theory such a type is expressed as $T^2$. The
    \lstinline|getUserByName| is now an embellished function and
    \lstinline|Maybe| is a monad, thus monad’s natural transformation $\mu : T^2
    \rightarrow T$ can be used to flatten the \lstinline|Maybe (Maybe User)|
    structure into a \lstinline|Maybe User|. In PureScript the natural
    transformation $\mu$ is called \lstinline|join|.
 
    \begin{lstlisting}[language=Haskell]
      user :: Maybe User
      user = join $ map getUserByName userInput
    \end{lstlisting} 

    Additionally a \lstinline|bind :: forall a b. m a -> (a -> m b) -> m b|,
    also known as \lstinline|>>=|, can be applied for more convenient syntax.
 
    \begin{lstlisting}[language=Haskell]
      user :: Maybe User
      user = userInput >>= getUserByName
    \end{lstlisting}

%------------------------------------------------------------

    \subsubsection{Eff and Aff Monad}
      PureScript obeys laws of functional programming and it does not have
      impure side-effects. In order to execute a side-effect, PureScript
      presents a concept called effects. Some examples of side-effects are
      console IO, random number generation, exceptions, local storage,
      XMLHttpRequest and DOM manipulation.

      \lstinline|Eff| monad enables pure execution context for side-effects.
      With the help of effect monad the function \lstinline|getUserByName| can now
      be rewritten to return the user data from local storage. The definition of
      \lstinline|getUserByName| is updated to

      \begin{lstlisting}[language=Haskell]
        getUserByName :: forall eff. String 
                        -> Eff (dom :: DOM | eff) (Maybe User)
      \end{lstlisting}
 
      \lstinline|Eff| type takes two type parameters from which the first is a
      row of effects and the second describes the return type of the
      \lstinline|Eff| monad. Here \lstinline|Eff| monad encapsulates side-effect
      interaction with \lstinline|localStorage| which is defined as a part of
      a DOM effect. As a result of the side-effect computation the function
      \lstinline|getUserByName| returns \lstinline|Maybe User| just as before.
 
      \lstinline|Eff| monad enables the modeling of side effects in type level.
      As a benefit, used side-effects can be restricted in type level and
      side-effectful computations can be easily distinguished from pure
      computations. In addition, \lstinline|Eff| has all the benefits of a monad
      category.
 
      Some of the side-effects have an asynchronous nature. A common way to handle
      an asynchronous computation is to introduce success and failure callbacks.
 
      \begin{lstlisting}[language=Haskell]
        getUserByName :: forall eff. String 
                        -> (Error -> Eff (dom :: DOM | eff) Unit) 
                        -> (User -> Eff (dom :: DOM | eff) Unit)
                        -> Eff (dom :: DOM | eff) Unit
      \end{lstlisting}
 
      However, this method easily leads into situation called callback hell when
      composing multiple asynchronous effects. Composition of asynchronous
      effects can be solved with a continuation monad transformer called
      \lstinline|ContT|. With the help of \lstinline|ContT|, \lstinline|Eff|
      monad can be extended to \lstinline|Async| type.
 
      \begin{lstlisting}[language=Haskell]
        type Async eff = ContT Unit (Eff eff)
      \end{lstlisting}

      Now the function \lstinline|getUserByName| can be rewrited to fetch the
      user over XmlHttpRequest.

      \begin{lstlisting}[language=Haskell]
        getUserByName :: forall eff. String 
                        -> Async (http :: HTTP | eff) (Either Error User)
      \end{lstlisting}
 
      As in case of \lstinline|Eff| monad, the operation of asynchronous
      side-effect is modeled in type level. In addition, \lstinline|Async|
      enables composition of monadic effects through continuation transformer. A
      popular library called purescript-aff introduces an alternative approach
      to asynchronous computation, which avoids the use of monad transformers to
      enable composition of async effects. An asynchronous effect monad called
      \lstinline|Aff| aims to eliminate the callback hell and mimic synchronous
      code with do notation, just as async keyword does in JavaScript.
 
      Just as Haskell, a PureScript application does nothing without a
      side-effect. The main function of an application is expecting an
      \lstinline|Eff| monad. All other effects of an application has to be
      composed with the main effect. That is why the ease of composition of
      effects is essential part of PureScript application.

%============================================================

\section{Discussion}

Category theory introduces a lot of reusable constructions to application
development. Several common constructions have theoretical base in category
theory. The benefits from well proven theories is indisputable except one could
argue that the learning curve is too steep. Type level reasoning of a program is
highly beneficial and usually confirms the correctness of the program. In
general, a program is composed of smaller programs. The composition of atomic
operations fits well into description of category theory, which is all about
morphisms and their composition.
 
Understanding categorical constructions is extremely beneficial to grasp the
full potential of statically typed functional programming language. Also an
ability to understand the correctness of a program results into high quality and
reliable programs. The toolset of category theory is extremely useful, but the
learning curve is really steep as the concepts of the theory are very abstract.
However, one does not need to know all the concepts of category theory before
achieving benefits from it. Also basic concepts of category theory are quite
self-contained and to use one of them usually does not require deep
understanding of an another concept. Higher abstraction concepts of category
theory, in turn, require knowledge of basic concepts progressively.
 
There exists a lot of studies about category theory, but there is a lack of
studies about practical usage of category theory in Web programming. PureScript
and its libraries are influenced by category theory. A further study might take
a more detailed look into libraries of PureScript and find how they are
influenced by category theory. For example, an user interface library called
Halogen relies a lot on categorical structures such as free, lenses, monads,
natural transformation, profunctors and functors. A further study might also
focus on basic and more advanced concepts of category theory that were not
mentioned in this paper.


%============================================================

\section{Conclusions}

Modern statically typed functional programming languages are highly influenced
by category theory. It has had a large impact on functional programming
language semantics along lambda calculus. Abstract properties of category theory
can be found all over PureScript. These constructions provide enhanced
reasoning and proven tools to everyday application development.
 
Categorical constructions, presented in this paper, are just a tip of a
iceberg of category theory. PureScript also relies on other categorical concepts
which were not discussed in this paper. A computer scientist is not required to
remember fine grained details of category theory in order to use functional
programming language that builds on top of category theory, but understanding
the concepts of category theory eases the learning curve.

%============================================================

\bibliographystyle{plain}
\bibliography{ct}

\end{document}